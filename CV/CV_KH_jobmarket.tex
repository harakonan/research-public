%% start of file
%% Copyright 2006-2015 Xavier Danaux (xdanaux@gmail.com).
%
% This work may be distributed and/or modified under the
% conditions of the LaTeX Project Public License version 1.3c,
% available at http://www.latex-project.org/lppl/.


\documentclass[11pt,a4paper,roman]{moderncv}        % possible options include font size ('10pt', '11pt' and '12pt'), paper size ('a4paper', 'letterpaper', 'a5paper', 'legalpaper', 'executivepaper' and 'landscape') and font family ('sans' and 'roman')

% moderncv themes
\moderncvstyle{classic}                             % style options are 'casual' (default), 'classic', 'banking', 'oldstyle' and 'fancy'
\moderncvcolor{blue}                               % color options 'black', 'blue' (default), 'burgundy', 'green', 'grey', 'orange', 'purple' and 'red'
%\renewcommand{\familydefault}{\sfdefault}         % to set the default font; use '\sfdefault' for the default sans serif font, '\rmdefault' for the default roman one, or any tex font name
%\nopagenumbers{}                                  % uncomment to suppress automatic page numbering for CVs longer than one page

% character encoding
%\usepackage[utf8]{inputenc}                       % if you are not using xelatex ou lualatex, replace by the encoding you are using
%\usepackage{CJKutf8}                              % if you need to use CJK to typeset your resume in Chinese, Japanese or Korean

\usepackage{harvard}
\cfoot{\thepage}

% adjust the page margins
\usepackage[scale=0.75]{geometry}
\sethintscolumntowidth{2024 Expected}
%\setlength{\hintscolumnwidth}{3cm}                % if you want to change the width of the column with the dates
%\setlength{\makecvheadnamewidth}{10cm}            % for the 'classic' style, if you want to force the width allocated to your name and avoid line breaks. be careful though, the length is normally calculated to avoid any overlap with your personal info; use this at your own typographical risks...

% personal data
\name{Konan}{Hara}
\title{Curriculum Vitae}                               % optional, remove / comment the line if not wanted
\address{1130 E. Helen Street}{McClelland Hall 401}{Tucson, AZ 85721-0108}% optional, remove / comment the line if not wanted; the "postcode city" and "country" arguments can be omitted or provided empty
\phone[mobile]{+1 (520) 475-9257}                   % optional, remove / comment the line if not wanted; the optional "type" of the phone can be "mobile" (default), "fixed" or "fax"
\phone[fixed]{+1 (520) 621-6224}
% \phone[fax]{+1 (520) 626-4623}
\email{harakonan@arizona.edu}                               % optional, remove / comment the line if not wanted
\homepage{harakonan.github.io}                         % optional, remove / comment the line if not wanted
%\social[linkedin]{john.doe}                        % optional, remove / comment the line if not wanted
%\social[xing]{john\_doe}                           % optional, remove / comment the line if not wanted
%\social[twitter]{jdoe}                             % optional, remove / comment the line if not wanted
\social[github]{harakonan}                              % optional, remove / comment the line if not wanted
%\social[gitlab]{jdoe}                              % optional, remove / comment the line if not wanted
%\social[skype]{jdoe}                               % optional, remove / comment the line if not wanted
\extrainfo{updated \today}                 % optional, remove / comment the line if not wanted
%\photo[64pt][0.4pt]{picture}                       % optional, remove / comment the line if not wanted; '64pt' is the height the picture must be resized to, 0.4pt is the thickness of the frame around it (put it to 0pt for no frame) and 'picture' is the name of the picture file
%\quote{Some quote}                                 % optional, remove / comment the line if not wanted

% bibliography adjustements (only useful if you make citations in your resume, or print a list of publications using BibTeX)
%   to show numerical labels in the bibliography (default is to show no labels)
% \makeatletter\renewcommand*{\bibliographyitemlabel}{\@biblabel{\arabic{enumiv}}}\makeatother
\renewcommand*{\bibliographyitemlabel}{[\arabic{enumiv}]}
%   to redefine the bibliography heading string ("Publications")
% \renewcommand{\refname}{Articles}

% bibliography with mutiple entries
\usepackage[resetlabels]{multibib}
\newcites{eco,medpri,medoth}{{Economics},{Public Health and Clinical Research (First/Corresponding Author)},{Public Health and Clinical Research (Others)}}
%----------------------------------------------------------------------------------
%            content
%----------------------------------------------------------------------------------

\newcommand{\cvreference}[7]{%
  \textbf{#1}\newline% Name
  \ifthenelse{\equal{#2}{}}{}{\addresssymbol~#2\newline}%
  \ifthenelse{\equal{#3}{}}{}{#3\newline}%
  \ifthenelse{\equal{#4}{}}{}{#4\newline}%
  \ifthenelse{\equal{#5}{}}{}{#5\newline}%
  \ifthenelse{\equal{#6}{}}{}{\emailsymbol~\texttt{#6}\newline}%
  \ifthenelse{\equal{#7}{}}{}{\phonesymbol~#7}}

\begin{document}
%\begin{CJK*}{UTF8}{gbsn}                          % to typeset your resume in Chinese using CJK
%-----       resume       ---------------------------------------------------------
\makecvtitle

\section{Education}
\cventry{2024 Expected}{Ph.D.}{Department of Economics}{University of Arizona}{}{}
\cventry{2019}{Ph.D.}{Department of Public Health}{The University of Tokyo}{}{}
\cventry{2013}{M.D. Equivalent}{}{The University of Tokyo}{}{A six-year undergraduate degree that allows one to obtain a medical license in Japan.}  % arguments 3 to 6 can be left empty
\cventry{2013}{B.A.}{Faculty of Medicine}{The University of Tokyo}{}{}

\section{Research Interests}
\cvitem{}{Energy and Environmental Economics; Industrial Organization; Econometrics; Health Economics}
% \cvlistitem{Energy and Environmental Economics}
% \cvlistitem{Industrial Organization}
% \cvlistitem{Econometrics}
% \cvlistitem{Health Economics}
% \cvlistitem{Biostatistics and Epidemiology}

% \section{Qualification}
% \cventry{2013}{Medical Doctor}{}{Japan}{}{}

\section{Appointments and Affiliations}

% \subsection{Academic Positions}
% \cventry{04/20 -- 01/21}{Project Researcher}{Department of Public Health}{The University of Tokyo}{Tokyo, Japan}{}
\cvitem{2019 -- 2020}{Researcher at TXP Medical Co., Ltd., Tokyo, Japan}
% \cventry{04/19 -- 08/20}{Researcher (Part-time)}{}{TXP Medical Co., Ltd.}{Tokyo, Japan}{}%{I am working on a project to create a framework that enables efficient usage of electronic health records for clinical research purposes. R is used for the project.}
\cvitem{2019 -- 2020}{Researcher at Department of Hematology, Tokyo Metropolitan Cancer and Infectious Diseases Center Komagome Hospital, Tokyo, Japan}
% \cventry{04/19 -- 12/20}{Researcher (Part-time)}{Department of Hematology}{Tokyo Metropolitan Cancer and Infectious Diseases Center Komagome Hospital}{Tokyo, Japan}{}%{I am working with clinical medicine specialists to apply statistical models that have rarely used in the domain but are considered to be useful theoretically to clinical data.}
% \cventry{04/19 -- 03/20}{Project Researcher}{Graduate School of Economics}{The University of Tokyo}{Tokyo, Japan}{}%{I was in charge of management and analysis of highly confidential microdata that covers the entire population in Japan: the Japanese national medical claims data and the Japanese Census data. The data size was up to several terabytes with 40 billion records per table. My main work was to perform a longitudinal linkage of individuals for each data. MySQL, Oracle SQL, R, Python, and Bash were used for the projects.}
% \cventry{07/17 -- 03/19}{Project Researcher (Part-time)}{Department of Public Health}{The University of Tokyo}{Tokyo, Japan}{}%{Using the Japanese national medical claims data, I described the trend in the share of generic drugs for commonly-prescribed medicines from 2009 to 2015, stratified by various demographic groups. Oracle SQL with Oracle Exadata was used for the project.}

% \subsection{Consulting and Advising}
% \cventry{04/21 -- }{Consultant}{Economic and Social Research Institute}{Cabinet Office, Government of Japan}{Tokyo, Japan}{}
% \cventry{08/20 -- }{Consultant}{}{TXP Medical Co., Ltd.}{Tokyo, Japan}{}

% \subsection{Appointments at Hospitals}
\cvitem{2015 -- 2020}{Nephrologist/Physician at several hospitals/clinics in Japan}
% \cventry{07/16 -- 12/20}{Nephrologist (Part-time)}{}{Chiba Aiyukai Kinen Hospital}{Chiba, Japan}{}
% \cventry{01/16 -- 06/16}{Nephrologist (Part-time)}{}{Ageo Central General Hospital Eight Nine Clinic}{Saitama, Japan}{}
% \cventry{04/15 -- 12/15}{Physician (Part-time)}{}{Nishimura Heart Clinic}{Saitama, Japan}{}
\cvitem{2013 -- 2015}{The University of Tokyo Hospital Junior Residency Program, Tokyo, Japan}
% \cventry{04/14 -- 03/15}{Intern}{}{The University of Tokyo Hospital}{Tokyo, Japan}{}
% \cventry{04/13 -- 03/14}{Intern}{}{The Fraternity Memorial Hospital}{Tokyo, Japan}{}

\section{Research Experience}
\cventry{2021 -- }{Research Assistant}{Department of Economics}{University of Arizona}{}{Energy and Environmental Economics research with Ashley Langer, Derek Lemoine, and Gautam Gowrisankaran}
% \cventry{05/21 -- }{Research Assistant for Ashley Langer}{}{University of Arizona}{}{Funded by: NSF grant (SES-2149335)}
% \cventry{08/21 -- 05/21}{Research Assistant for Ashley Langer and Derek Lemoine}{}{University of Arizona}{}{Funded by: Arizona Institutes for Resilience}
% \cventry{01/21 -- 07/21}{Research Assistant for Gautam Gowrisankaran}{}{University of Arizona}{}{Funded by: NSF grant (SES-1824348)}
\cventry{2016 -- 2020}{Research Assistant}{}{The University of Tokyo}{Tokyo, Japan}{Public Health/Health Economics research with Hidehiko Ichimura, Yasuki Kobayashi, and Yuichi Tei}
% \cventry{08/17 -- 03/19}{Research Assistant for Hidehiko Ichimura}{Graduate School of Economics}{The University of Tokyo}{}{Funded by: Japan Society for the Promotion of Science KAKENHI (15H05692)}%\\I worked on a project to investigate the preventive effects of early consultation on the onset of fatal diseases through the improvement of lifestyle diseases using large-scale medical claims data and annual health checkup data. Microsoft SQL Server and R were used for the project.}
% \cventry{02/17 -- 03/17}{Research Assistant for Hajime Sato}{}{National Institute of Public Health}{}{Funded by: Japanese Ministry of Health, Labour and Welfare (H29-tokubetsu-shitei-035)}%\\I was engaged in the survey and summarization of the current state of clinical research regulations in Japan for the establishment of the clinical research law.}
% \cventry{07/16 -- 03/17}{Research Assistant for Yuichi Tei}{Graduate School of Engineering}{The University of Tokyo}{}{Funded by: Japan Science and Technology Agency (JPMJCE1304)}%\\As a data manager, I worked on anonymization, data cleaning, and data extraction of company labor management systems and annual health checkup data. R was used for the project.}

\section{Teaching Experience}
% sole instructor of record
\cventry{2023}{Sole Instructor of Record}{Department of Economics}{University of Arizona}{}{Industrial Organization (Summer 2023)}
\cventry{2021 -- }{Teaching Assistant}{Department of Economics}{University of Arizona}{}{
Intermediate Microeconomics (Fall 2023)\\
PhD Econometrics (Fall 2021, Spring 2022, Fall 2022, Spring 2023)\\
Health Economics (Spring 2021)
}
% \cventry{Fall 2022}{Teaching Assistant for Hidehiko Ichimura and James Powell}{}{University of Arizona}{}{ECON 520: Probability/Statistics and Econometrics for 1st-year PhD students (core)}
% \cventry{Spring 2022}{Teaching Assistant for Hidehiko Ichimura}{}{University of Arizona}{}{ECON 522A: Econometrics for 1st-year PhD students (core)}
% \cventry{Fall 2021}{Teaching Assistant for Tiemen Woutersen}{}{University of Arizona}{}{ECON 522B: Econometrics for 2nd-year PhD students (core)}
% \cventry{Spring 2021}{Teaching Assistant for Juan Pantano}{}{University of Arizona}{}{ECON 422: Health Economics for advanced undergraduate students}

% \section{Research Grants}
% \cventry{2019}{Investigator}{Elucidation of regulatory mechanisms for latent infection in progenitor cells and molecular mechanisms for virus reactivation along with cell differentiation (PI: Ayumi Taguchi)}{Japan Agency for Medical Research and Development (AMED)}{}{}%{We conducted two projects that explore the application of statistical models that have rarely used in the domain but are considered to be useful theoretically to clinical data: (1) we applied a statistical method for panel data from a semi-Markov process to predict the prognosis of cervical intraepithelial neoplasia according to the genotypes of high-risk human papillomavirus; (2) we applied survival analysis considering competing risks, time-dependent covariates, and interaction terms for exploring the heterogeneous impact of cytomegalovirus reactivation on non-relapse mortality in hematopoietic stem cell transplantation.}

% \section{Dissertation}
% \cvitem{Title}{\emph{Claims-based algorithms for common chronic conditions were investigated using regularly collected data in Japan}}
% \cvitem{Supervisor}{Yasuki Kobayashi}
% \cvitem{Description}{The literature of claims-based algorithm (CBA) has two features to be refined: the use of a chart review as a source of the gold standard; the procedure of searching for a fine-tuned CBA based on existing knowledge regarding target conditions. The first feature limits the population to which the CBA can be applied and the second makes the CBA construction procedure to be an overly complicated and cumbersome matter. Moreover, the burden of reviewing charts and searching for a fine-tuned CBA lead to a slow establishment of acceptable CBAs because it discourages researchers from CBA studies. The sluggish establishment of usable CBAs can be a big issue as the codes recorded in the claims for transmitting information about patients are supposed to change periodically. The dissertation focused on CBAs for identifying patients with three common chronic medical conditions, hypertension, diabetes, and dyslipidemia, and (1) demonstrated the usefulness of health screening results as the source of gold standard; (2) showed the power of statistical learning methods to develop an efficient CBA construction procedure; (3) proposed a course of action for an efficient CBA research. I believe that the series of techniques evaluated in the study should become essential in future CBA research.}

\section{Working Papers}

\cvitem{}{``Encouraging Renewable Investment: Risk Sharing Using Auctions'' (Job Market Paper)}

\section{Work in Progress}

\cvitem{}{``Estimating Dynamic Games with Unknown Information Structure'' (with Yuki Ito [\href{https://www.econ.berkeley.edu/grad/profiles/15341}{\underline{link}}] and Paul Koh [\href{https://www.pskoh.com/}{\underline{link}}])}
\cvitem{}{``Primary Care Physician-Specialist Racial Concordance in Forming Referral Networks'' (with Yuki Ito [\href{https://www.econ.berkeley.edu/grad/profiles/15341}{\underline{link}}])}
\cvitem{}{``Regulating Power Plant Emissions: Environmental Justice, Enforcement, and Regulator Preferences'' (with Gautam Gowrisankaran and Ashley Langer)}


% Publications from a BibTeX file without multibib
% not using this way here

% update bbl file after the update of publications.bib
% \nocite{Ito2019,Hara2018,Kinoshita2017} % comment out after creating bbl file
% \bibliographystyle{unsrt}
% \bibliography{publications}                       % 'publications' is the name of a BibTeX file

% Publications from a BibTeX file using the multibib package

% update publications.bib
% update citations in \nociteeco, \nocitemedpri, and \nocitemedoth by activating the following three lines:
% \nociteeco{Ito2020}
% \nocitemedpri{Ikesu2022,Hara2021,Hara2018}
% \nocitemedoth{Nara2023,Taguchi2023,Toshimitsu2023,Kishida2022,Fujimori2022,Kosaka2021,Ito2021,Osawa2021,Fukaguchi2021,Soeno2021,Goto2020,Kinoshita2020,Shimizuguchi2020,Kaito2020,Baba2020,Taguchi2020b,Taguchi2020a,Ito2019,Kinoshita2017}
% compile this tex file (make sure to not comment out \bibliography commands below)
% comment out \nociteeco, \nocitemedpri, and \nocitemedoth
% ./prepare_bbl.sh
% compile this tex file again

% 'publications' is the name of a BibTeX file
\section{Publications}
% \bibliographystyleeco{unsrt}
% \bibliographyeco{publications}                   
% \bibliographystylemedpri{unsrt}
% \bibliographymedpri{publications}

% explicitly write out to customize the item numbers
\subsection{Economics}
\cvitem{[1]}{
Yuki Ito, \textbf{\underline{Konan Hara}}, and Yasuki Kobayashi.
\newblock {The Effect of Inertia on Brand-Name versus Generic Drug Choices}.
\newblock {\em Journal of Economic Behavior {\&} Organization}, 172:364--379,
  2020.
}

\subsection{Public Health and Clinical Research (First/Corresponding Author)}
\cvitem{[2]}{
Ryo Ikesu, Ayumi Taguchi, \textbf{\underline{Konan Hara}}, Kei Kawana, Tetsushi Tsuruga, Jun Tomio,
  and Yutaka Osuga.
\newblock {Prognosis of High‐Risk Human Papillomavirus‐Related Cervical
  Lesions: A Hidden Markov Model Analysis of a Single‐Center Cohort in
  Japan}.
\newblock {\em Cancer Medicine}, 11(3):664--675, 2022.
}
\cvitem{[3]}{
\textbf{\underline{Konan Hara}}, Yasuki Kobayashi, Jun Tomio, Yuki Ito, Thomas Svensson, Ryo Ikesu,
  Ung-il Chung, and Akiko~Kishi Svensson.
\newblock {Claims-Based Algorithms for Common Chronic Conditions Were
  Efficiently Constructed Using Machine Learning Methods}.
\newblock {\em PLOS ONE}, 16(9):e0254394, 2021.
}
\cvitem{[4]}{
\textbf{\underline{Konan Hara}}, Jun Tomio, Thomas Svensson, Rika Ohkuma, Akiko~Kishi Svensson, and
  Tsutomu Yamazaki.
\newblock {Association Measures of Claims-Based Algorithms for Common Chronic
  Conditions Were Assessed Using Regularly Collected Data in Japan}.
\newblock {\em Journal of Clinical Epidemiology}, 99:84--95, 2018.
}

% \bibliographystylemedoth{unsrt}
% \bibliographymedoth{publications}
% short version for "other" research
\subsection{Public Health and Clinical Research (Others)}
\cvitem{[5--23]}{19 other publications, including in \textit{Cancer}, \textit{Blood Advances}, and \textit{American Journal of Emergency Medicine}}

\section{Conference Presentations}
\cvitem{2023}{Arizona Workshop on Environment, Natural Resource, and Energy Economics}
\cvitem{2020}{Machine Learning for Healthcare}
\cvitem{2019}{Annual Meeting of the Japanese Association for Acute Medicine}
\cvitem{2017}{International Health Economics Association 12th World Congress; Japanese Society of Public Health Annual Meeting}
\cvitem{2015}{Japanese Society of Public Health Annual Meeting}

\section{Honors and Awards}
\cventry{2022}{Ed Zajac Prize}{}{University of Arizona}{}{Best 3rd-year paper award, Paper title: Dynamic Games with Weak Assumptions on Information}
\cventry{2021}{Steve Manos Prize}{}{University of Arizona}{}{Best 2nd-year paper award, Paper title: The Role of Real Estate Office/Agent in Racial Residential Location Differences}
\cventry{2021}{Eller Graduate General Scholarship}{}{University of Arizona}{}{Award of excellence for 2nd-year class performance}
\cventry{2021 -- 2023}{Rotary Foundation District Grant Scholarship}{}{Rotary International District 2580, Rotary Foundation}{}{}
\cventry{2014}{Award of Excellence}{Internal Medicine Lecture Series for Junior Residents}{The University of Tokyo Hospital}{}{}

\section{Computer Skills}
\cvitem{Data Science}{Julia, R, Python, Stata}
\cvitem{RDBMS/DWH}{MySQL, Oracle SQL, SQL Server, Oracle Exadata}
\cvitem{Others}{LaTeX, GFM, R markdown, Zsh, Bash}
% \cvitem{Data Science}{R (Advanced), Python (Intermediate), Stata (Novice)}
% \cvitem{RDBMS}{MySQL (Advanced), Oracle SQL (Intermediate), SQL Server (Novice)}
% \cvitem{DWH}{Oracle Exadata (Intermediate)}
% \cvitem{Documentation}{LaTeX (Advanced), GFM (Advanced), R markdown (Intermediate)}
% \cvitem{Others}{Zsh (Advanced), Bash (Advanced)}
% \cvitem{}{Advanced $=$ Regularly used, Intermediate $=$ Occasionally used, Novice $=$ Rarely used}

% \section{Personal Information}
% \cvitem{Citizenship}{Japan}
% \cvitem{Language}{English (fluent), Japanese (native)}

\section{References}
\begin{cvcolumns}
\cvcolumn{}
{\cvreference
{Prof. Ashley Langer}
{University of Arizona}
{Department of Economics}
{}
{}
{alanger@arizona.edu}
{}
}
\cvcolumn{}
{\cvreference
{Prof. Hidehiko Ichimura}
{University of Arizona}
{Department of Economics}
{}
{}
{ichimura@arizona.edu}
{}
}
\end{cvcolumns}

\begin{cvcolumns}
\cvcolumn{}
{\cvreference
{Prof. Derek Lemoine}
{University of Arizona}
{Department of Economics}
{}
{}
{dlemoine@arizona.edu}
{}
}
\cvcolumn{}
{\cvreference
{Prof. Matthijs Wildenbeest}
{University of Arizona}
{Department of Economics}
{}
{}
{wildenbeest@arizona.edu}
{}
}
\end{cvcolumns}

\begin{cvcolumns}
\cvcolumn{Teaching Reference}
{\cvreference
{Prof. John Drabicki}
{University of Arizona}
{Department of Economics}
{}
{}
{drabicki@arizona.edu}
{}
}
\end{cvcolumns}

\end{document}

%% end of file