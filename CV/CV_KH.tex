%% start of file
%% Copyright 2006-2015 Xavier Danaux (xdanaux@gmail.com).
%
% This work may be distributed and/or modified under the
% conditions of the LaTeX Project Public License version 1.3c,
% available at http://www.latex-project.org/lppl/.


\documentclass[11pt,a4paper,roman]{moderncv}        % possible options include font size ('10pt', '11pt' and '12pt'), paper size ('a4paper', 'letterpaper', 'a5paper', 'legalpaper', 'executivepaper' and 'landscape') and font family ('sans' and 'roman')

% moderncv themes
\moderncvstyle{classic}                             % style options are 'casual' (default), 'classic', 'banking', 'oldstyle' and 'fancy'
\moderncvcolor{blue}                               % color options 'black', 'blue' (default), 'burgundy', 'green', 'grey', 'orange', 'purple' and 'red'
%\renewcommand{\familydefault}{\sfdefault}         % to set the default font; use '\sfdefault' for the default sans serif font, '\rmdefault' for the default roman one, or any tex font name
%\nopagenumbers{}                                  % uncomment to suppress automatic page numbering for CVs longer than one page

% character encoding
%\usepackage[utf8]{inputenc}                       % if you are not using xelatex ou lualatex, replace by the encoding you are using
%\usepackage{CJKutf8}                              % if you need to use CJK to typeset your resume in Chinese, Japanese or Korean

\usepackage{harvard}
\cfoot{\thepage}

% adjust the page margins
\usepackage[scale=0.75]{geometry}
\sethintscolumntowidth{04/19 -- present}
%\setlength{\hintscolumnwidth}{3cm}                % if you want to change the width of the column with the dates
%\setlength{\makecvheadnamewidth}{10cm}            % for the 'classic' style, if you want to force the width allocated to your name and avoid line breaks. be careful though, the length is normally calculated to avoid any overlap with your personal info; use this at your own typographical risks...

% personal data
\name{Konan}{Hara}
\title{Curriculum Vitae}                               % optional, remove / comment the line if not wanted
\address{7-3-1 Hongo, Bunkyo-ku}{Tokyo 113-0033}{Japan}% optional, remove / comment the line if not wanted; the "postcode city" and "country" arguments can be omitted or provided empty
\phone[mobile]{+81 90-8507-1062}                   % optional, remove / comment the line if not wanted; the optional "type" of the phone can be "mobile" (default), "fixed" or "fax"
\phone[fixed]{+81 3-5841-5543}
%\phone[fax]{+81 3-5841-3303}
\email{hara.konan@gmail.com}                               % optional, remove / comment the line if not wanted
%\homepage{www.johndoe.com}                         % optional, remove / comment the line if not wanted
%\social[linkedin]{john.doe}                        % optional, remove / comment the line if not wanted
%\social[xing]{john\_doe}                           % optional, remove / comment the line if not wanted
%\social[twitter]{jdoe}                             % optional, remove / comment the line if not wanted
\social[github]{harakonan}                              % optional, remove / comment the line if not wanted
%\social[gitlab]{jdoe}                              % optional, remove / comment the line if not wanted
%\social[skype]{jdoe}                               % optional, remove / comment the line if not wanted
\extrainfo{updated \today}                 % optional, remove / comment the line if not wanted
%\photo[64pt][0.4pt]{picture}                       % optional, remove / comment the line if not wanted; '64pt' is the height the picture must be resized to, 0.4pt is the thickness of the frame around it (put it to 0pt for no frame) and 'picture' is the name of the picture file
%\quote{Some quote}                                 % optional, remove / comment the line if not wanted

% bibliography adjustements (only useful if you make citations in your resume, or print a list of publications using BibTeX)
%   to show numerical labels in the bibliography (default is to show no labels)
% \makeatletter\renewcommand*{\bibliographyitemlabel}{\@biblabel{\arabic{enumiv}}}\makeatother
\renewcommand*{\bibliographyitemlabel}{[\arabic{enumiv}]}
%   to redefine the bibliography heading string ("Publications")
% \renewcommand{\refname}{Articles}

% bibliography with mutiple entries
\usepackage[resetlabels]{multibib}
\newcites{eco,med}{{Economics},{Public Health and Clinical Research}}
%----------------------------------------------------------------------------------
%            content
%----------------------------------------------------------------------------------
\begin{document}
%\begin{CJK*}{UTF8}{gbsn}                          % to typeset your resume in Chinese using CJK
%-----       resume       ---------------------------------------------------------
\makecvtitle

\section{Personal Information}
\cvitem{Citizenship}{Japan}
\cvitem{Language}{English (fluent), Japanese (native)}

\section{Education}
\cventry{2019}{Ph.D. in Medicine}{Graduate School of Medicine, The University of Tokyo}{Japan}{}{}
\cventry{2013}{M.D. Equivalent}{Faculty of Medicine, The University of Tokyo}{Japan}{}{}  % arguments 3 to 6 can be left empty
\cventry{2013}{B.A.}{Faculty of Medicine, The University of Tokyo}{Japan}{}{}

\section{Research Interests}
\cvlistitem{Econometrics}
\cvlistitem{Biostatistics and Epidemiology}
\cvlistitem{Health Economics}
\cvlistitem{Labor Economics}
\cvlistitem{Public Finance}

\section{Qualification}
\cventry{2013}{Medical Doctor}{Japan}{}{}{}

\section{Appointments and Affiliations}

\subsection{Academic Positions}
\cventry{04/20 -- present}{Project Researcher}{Department of Public Health, Graduate School of Medicine, The University of Tokyo}{Tokyo}{Japan}{}
\cventry{04/19 -- present}{Part-time Researcher}{TXP Medical Co., Ltd.}{Tokyo}{Japan}{I am working on a project to create a framework that enables efficient usage of electronic health records for clinical research purposes. R is used for the project.}
\cventry{04/19 -- present}{Part-time Researcher}{Department of Hematology, Tokyo Metropolitan Cancer and Infectious Diseases Center Komagome Hospital}{Tokyo}{Japan}{I am working with clinical medicine specialists to apply statistical models that have rarely used in the domain but are considered to be useful theoretically to clinical data.}
\cventry{04/19 -- 03/20}{Project Researcher}{Graduate School of Economics, The University of Tokyo}{Tokyo}{Japan}{I was in charge of management and analysis of highly confidential microdata that covers the entire population in Japan: the Japanese national medical claims data and the Japanese Census data. The data size was up to several terabytes with 40 billion records per table. My main work was to perform a longitudinal linkage of individuals for each data. MySQL, Oracle SQL, R, Python, and Bash were used for the projects.}
\cventry{07/17 -- 03/19}{Project Researcher}{Department of Public Health, Graduate School of Medicine, The University of Tokyo}{Tokyo}{Japan}{Using the Japanese national medical claims data, I described the trend in the share of generic drugs for commonly-prescribed medicines from 2009 to 2015, stratified by various demographic groups. Oracle SQL with Oracle Exadata was used for the project.}

\subsection{Appointments at Hospitals}
\cventry{07/16 -- present}{Nephrologist}{Chiba Aiyukai Kinen Hospital}{Chiba}{Japan}{}
\cventry{01/16 -- 06/16}{Nephrologist}{Ageo Central General Hospital Eight Nine Clinic}{Saitama}{Japan}{}
\cventry{04/15 -- 12/15}{Physician}{Nishimura Heart Clinic}{Saitama}{Japan}{}
\cventry{04/14 -- 03/15}{Intern}{The University of Tokyo Hospital}{Tokyo}{Japan}{}
\cventry{04/13 -- 03/14}{Intern}{The Fraternity Memorial Hospital}{Tokyo}{Japan}{}

\section{Research Experience}
\cventry{08/17 -- 03/19}{Research Assistant for Hidehiko Ichimura}{Graduate School of Economics, The University of Tokyo}{}{}{I worked on a project to investigate the preventive effects of early consultation on the onset of fatal diseases through the improvement of lifestyle diseases using large-scale medical claims data and annual health checkup data. Microsoft SQL Server and R were used for the project.}
\cventry{02/17 -- 03/17}{Research Assistant for Hajime Sato}{National Institute of Public Health}{}{}{I was engaged in the survey and summarization of the current state of clinical research regulations in Japan for the establishment of the clinical research law.}
\cventry{07/16 -- 03/17}{Research Assistant for Yuichi Tei}{Graduate School of Engineering, The University of Tokyo}{}{}{As a data manager, I worked on anonymization, data cleaning, and data extraction of company labor management systems and annual health checkup data. R was used for the project.}

\section{Honors and Awards}
\cventry{2014}{Award of Excellence}{Grand Round of Internal Medicine, The University of Tokyo Hospital}{}{}{}

\section{Grants}
\cventry{2019}{Investigator}{Elucidation of regulatory mechanisms for latent infection in progenitor cells and molecular mechanisms for virus reactivation along with cell differentiation (PI: Ayumi Taguchi)}{Japan Agency for Medical Research and Development (AMED)}{}{We conducted two projects that explore the application of statistical models that have rarely used in the domain but are considered to be useful theoretically to clinical data: (1) we applied a statistical method for panel data from a semi-Markov process to predict the prognosis of cervical intraepithelial neoplasia according to the genotypes of high-risk human papillomavirus; (2) we applied survival analysis considering competing risks, time-dependent covariates, and interaction terms for exploring the heterogeneous impact of cytomegalovirus reactivation on non-relapse mortality in hematopoietic stem cell transplantation.}

\section{Dissertation}
\cvitem{Title}{\emph{Claims-based algorithms for common chronic conditions were investigated using regularly collected data in Japan}}
\cvitem{Supervisor}{Yasuki Kobayashi}
\cvitem{Description}{The literature of claims-based algorithm (CBA) has two features to be refined: the use of a chart review as a source of the gold standard; the procedure of searching for a fine-tuned CBA based on existing knowledge regarding target conditions. The first feature limits the population to which the CBA can be applied and the second makes the CBA construction procedure to be an overly complicated and cumbersome matter. Moreover, the burden of reviewing charts and searching for a fine-tuned CBA lead to a slow establishment of acceptable CBAs because it discourages researchers from CBA studies. The sluggish establishment of usable CBAs can be a big issue as the codes recorded in the claims for transmitting information about patients are supposed to change periodically. The dissertation focused on CBAs for identifying patients with three common chronic medical conditions, hypertension, diabetes, and dyslipidemia, and (1) demonstrated the usefulness of health screening results as the source of gold standard; (2) showed the power of statistical learning methods to develop an efficient CBA construction procedure; (3) proposed a course of action for an efficient CBA research. I believe that the series of techniques evaluated in the study should become essential in future CBA research.}

% Publications from a BibTeX file without multibib

% update bbl file after the update of publications.bib
% \nocite{Ito2019,Hara2018,Kinoshita2017} % comment out after creating bbl file
% \bibliographystyle{unsrt}
% \bibliography{publications}                       % 'publications' is the name of a BibTeX file

% Publications from a BibTeX file using the multibib package

% update publications.bib
% update citations in \nociteeco and \nocitemed:
% \nociteeco{Ito2020}
% \nocitemed{Kaito2020,Baba2020,Taguchi2020b,Taguchi2020a,Ito2019,Hara2018,Kinoshita2017}
% update bbl file (compile this tex file)
% comment out \nociteeco and \nocitemed
% ./prepare_bbl.sh
% compile this tex file again

% 'publications' is the name of a BibTeX file
\section{Publications}
\bibliographystyleeco{unsrt}
\bibliographyeco{publications}                   
\bibliographystylemed{unsrt}
\bibliographymed{publications}

\section{Working Papers}

\subsection{Economics}
\cvitem{}{NULL}

\subsection{Public Health and Clinical Research}
\cvitem{[1]}{Yuki Kosaka, Takehiro Sugiyama, \textbf{\underline{Konan Hara}}, and Yasuki Kobayashi. Investigating the adherence to daily, weekly, and monthly bisphosphonates for women with postmenopausal osteoporosis using claims data in Japan. 2020.}
\cvitem{[2]}{Takuya Shimizuguchi, Noritaka Sekiya, \textbf{\underline{Konan Hara}}, Ayumi Taguchi, Yujiro Nakajima, Yu Miyake, Yukiko Shibata, Kentaro Taguchi, Hiroaki Ogawa, Kei Ito, Keiji Nihei, and Katsuyuki Karasawa. Radiation therapy and risk of herpes zoster in patients with cancer. 2020. \emph{Revision Requested, Cancer}.}

\section{Work in Progress}

\subsection{Economics}
\cvitem{[1]}{Taiyo Fukai, \textbf{\underline{Konan Hara}}, Hidehiko Ichimura, and Kyogo Kanazawa. Estimating the long-term transition of medical expenditures using the Japanese national claims database.}
\cvitem{[2]}{Taiyo Fukai, \textbf{\underline{Konan Hara}}, and Hidehiko Ichimura. What do we learn from Panel Data Created Using Japanese Census?}
\cvitem{[3]}{\textbf{\underline{Konan Hara}}, Hidehiko Ichimura, and Yuki Ito. Estimating the effect of early consultation on the onset of fatal diseases.}

\subsection{Public Health and Clinical Research}
\cvitem{[1]}{Hidehiko Ichimura, \textbf{\underline{Konan Hara}}, Taiyo Fukai, Atsushi Miyawaki, Kazuhiro Abe, Kyogo Kanazawa, Haruko Noguchi, Nanako Tamiya, and Yasuki Kobayashi. End-of-Life Medical and Long-term Care Spending in Japan.}
\cvitem{[2]}{\textbf{\underline{Konan Hara}}, Ryoya Yoshihara, Tomohiro Sonoo, Toru Shirakawa, Tadahiro Goto, and Kensuke Nakamura. Development of phenotype algorithms for common acute conditions using SHapley Additive exPlanation values.}
\cvitem{[3]}{\textbf{\underline{Konan Hara}}, Yasuki Kobayashi, Yuki Ito, Jun Tomio, Thomas Svensson, and Akiko Kishi Svensson. Development of an efficient claims-based algorithm construction procedure using machine learning methods.}
\cvitem{[4]}{Yuki Ito, \textbf{\underline{Konan Hara}}, and Jun Tomio. Perception of Generic Drugs and Willingness-to-pay: A Survey-based Study.}

\section{Conference Presentations}
\cvitem{2019}{Annual Meeting of the Japanese Association for Acute Medicine}
\cvitem{2017}{International Health Economics Association 12th World Congress; Japanese Society of Public Health Annual Meeting}
\cvitem{2015}{Japanese Society of Public Health Annual Meeting}

\section{Computer Skills}
\cvitem{Data Science}{R (Advanced), Python (Intermediate), Stata (Novice)}
\cvitem{RDBMS}{MySQL (Advanced), Oracle SQL (Intermediate), SQL Server (Novice)}
\cvitem{DWH}{Oracle Exadata (Intermediate)}
\cvitem{Documentation}{LaTeX (Advanced), GFM (Advanced), R markdown (Intermediate)}
\cvitem{Others}{Zsh (Advanced), Bash (Advanced)}
\cvitem{}{Advanced $=$ Regularly used, Intermediate $=$ Occasionally used, Novice $=$ Rarely used}

\end{document}

%% end of file